\documentclass[a4paper,12pt,twocolumn]{article}

\usepackage[english]{babel}
\usepackage[T1]{fontenc}
\usepackage[utf8]{inputenc}
\usepackage{amsmath}
\usepackage[pdftex]{graphicx}
\usepackage{xcolor}
\usepackage{fourier}
\usepackage[protrusion=true,expansion=true]{microtype}
\usepackage[colorinlistoftodos]{todonotes}
\usepackage{fancyhdr}
\usepackage{enumitem}
\usepackage{listings}
\lstset{basicstyle=\footnotesize\ttfamily,breaklines=true}
\lstset{frame=tb,language=C,numbers=left,showstringspaces=false}
\renewcommand{\lstlistingname}{Code Block}

\usepackage{geometry}
\geometry{total={210mm,297mm},
left=25mm, right=25mm,
bindingoffset=0mm,
top=20mm, bottom=20mm}

\usepackage[
  pdftitle={Nearest Neighbor Search},
  pdfauthor={William Jagels, Rushil Kumar},
  colorlinks=true,linkcolor=blue,urlcolor=blue,
  citecolor=blue,bookmarks=true,bookmarksopenlevel=2
]{hyperref}

\usepackage{titlesec}
\titlelabel{\thetitle.\quad}

\def\code#1{\texttt{#1}}

\title{Nearest Neighbor Search}

\author{William Jagels and Rushil Kumar}

\date{\today}

\begin{document}
\maketitle

\begin{abstract}
  In this paper, we set out to research and implement algorithms for Nearest Neighbor Search (NNS).
  Our implementation is able to make point clouds with arbitrary dimensional points.
  The nearest neighbor is also suited to work with these point clouds at any dimension.
  We discuss our testing results and compare our implementation to other common implementations
\end{abstract}

\section{Problem}
Given a target point, and a list of other points, find the closest point to the target.
This problem appears in the everyday world, such as finding the nearest stores on a
retailer's website or finding the closest airport for an aircraft in distress.
Although many appearances of this problem are trivially solved by a linear search,
sometimes a more efficient algorithm is needed.

\section{k-Nearest Neighbors}
We can observe that with the regular NNS problem, it is impossible to get any better than
$\Theta(N)$ time unless we can make assumptions about the ordering of our point cloud.
In the k-NNS problem, where we find the closest $k$ neighbors, our naïve linear search
will degenerate into a $\Theta(kN)$ time complexity algorithm.
At larger values of $k$, it becomes clear that a linear search is no longer reasonable.

\section{k-Dimensional Tree}
One data structure commonly used to solve the NNS problem is a k-dimensional(k-d) tree.
A k-d tree is a binary search tree that is sorted in multiple dimensions, allowing
for a $O(n \log n)$ search time.
This data structure uses space partitioning to achieve its performance characteristics.
At each level of a k-d tree, there is a specified axis that the left and right subtrees
are divided by.
We implemented and used a k-d tree specifically for nearest neighbor search, but it can
also be used for problems such as range searching.

\section{k-d Tree NNS}
We implemented a nearest neighbor search that works for our k-d tree implementation,
and found that it could reliably find the nearest neighbor to a point very quickly.
% TODO: write more

\section{Results}
Using $3*10^6$ points in 3 dimensions, linear search takes on the order of $10^{-2}$s.
On the same test, k-d tree construction takes on the order of $10^0$s, and NNS takes
on the order of $10^{-6}$s.
It is important to note that the k-d tree construction only has to happen once per point
cloud, and any additional nearest neighbor searches will take the same amount of time.
Since the k-d tree construction is slower by a factor of $10^2$, if we perform any more
than $10^2$ searches, we should see a performance gain over the linear search.
$10^2$ searches may seem large, however, that is $0.0033\%$ of the size of the point cloud.

\section{Analysis}
Our tests revealed that our NNS algorithm is very fast on an already created k-d tree, but
the k-d tree construction is rather inefficient.
Rather than attempt to optimize our own k-d tree (reinventing the wheel), we adapted
our code to run on more popular k-d tree implementations.

\section{Further Testing}
% TODO: write this part

\end{document}
